Wireless sensor networks are used to increase the
efficiency of many applications, such as target detection and
disaster management. Wireless sensor networks with static
nodes have been developed and also experimentally applied
for detection and monitoring activities [1]. However, static
wireless sensor networks have important limitations as far as
the required coverage and the short communication range in
the nodes are concerned. The use of mobile nodes could
provide significant improvements. Thus, they can provide the
ability to dynamically adapt the network to environmental
events and to improve the network connectivity in case of
static nodes failure. On the other hand, the aerial and
remotely piloted vehicles are now able to be coordinated for
missions such as the detection and monitoring of events.

\cite{Aware}
%%%%%%%%%%%%%%%%%%%%%%%%%%%%%%%%%%%%%%%%%%%%%%%%%%%%%%%%%%%%%
The capabilities and roles of Unmanned Aerial Vehicles
(UAVs) are evolving, and require new concepts
for their control.1 For example, today’s UAVs typically
require several operators, but future UAVs will
be designed to make tactical decisions autonomously
and will be integrated into teams that coordinate to
achieve high-level goals, thereby allowing one operator
to control a group of vehicles.

Arthur Richards
%%%%%%%%%%%%%%%%%%%%%%%%%%%%%%%%%%%%%%%%%%%%%%%%%%%%%%%%%%%%%%
A static sink is normally located
around the boundary of the network; therefore, it
usually requires many round trips over the network in
order to complete one transaction. This results in poor
performance of the system, inefficient use of energy
and of available bandwidth.

Shih-Hao Chang



%%%%%%%%%%%%%%%%%%%%%%%%%%%%%%%%%%%%%%%%%%%%%%%%%%%%%%%%%%%%%%%%

An interesting scenario that is receiving great attention
consists of sensors whose information must be transmitted toward a far receiver located onboard of unmanned
aerial vehicles (UAVs) [4],

We consider a WSN with N nodes deployed randomly over a wide planar area (Fig. 1). These nodes can collect
environmental data such as temperature, pressure, humidity, etc. and transmit them to the fusion center onboard
the UAV whenever they receive a message triggered from the vehicle. As previously stated, we assume that all
nodes share the same data.

Luchi






%%%%%%%%%%%%%%%%%%%%%%%%%%%%%%%%%%%%%%%%%%%%%%%%%%%%%%%%%%%%

The ability to actively change the
location of sensors can be used to mitigate some
of the traditional problems associated with static
sensor networks. On the other hand, sensor
mobility brings its own challenges. These include
challenges associated with in-network aggregation
of sensor data, routing, and activity monitoring
of responders. Moreover, all different
mobility patterns (e.g., sink mobility, sensor
mobility) have their special properties, so that
each mobile device class needs its own approach.
In this article, we present a platform which benefits
from both static and mobile sensors and
addresses these challenges. 

The main objective of the sensingactuating
system is to detect events (e.g., fire)
by means of sensors and wirelessly communicate
this event and assist other nodes to deliver
the event.

However, static WSNs have some limitations.
The use of mobile sensors could provide
significant improvements. They can provide the
ability to closely monitor the objects we want
to guard in WSNs and look at the events at a
smaller granularity than static nodes. Also,
multiple mobile sinks inside the monitored
region can collect the data from the sensors
when they pass by. Introducing multiple mobile
sinks in WSNs can provide fast energy-efficient
data collection with well designed networking
protocols.

Mobility of sensor and sink nodes can be
achieved by some vehicles or people carrying
sensors. It is more efficient to use vehicles
instead of people in some cases like disaster
management applications due to harsh environmental
conditions during a disaster. For this purpose,
using aerial and remotely piloted vehicles
is a promising idea.

(aware 2)