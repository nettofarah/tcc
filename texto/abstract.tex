\abstract{
 Este trabalho apresenta uma investigação de estratégias para coordenar um conjunto de nós sensores terrestres estáticos (posicionados no solo) e de Veículos Aéreos Não Tripulados (UAVs) que carregam uma variedade de sensores. Esta coordenação tem como objetivo prover monitoramento e detecção eficientes de intrusos em uma determinada área de interesse. Para o desenvolvimento desta coordenação tem-se o uso de técnicas de Auto-Organização Emergente. Estas técnicas não utilizam controles externos ou centralizados, contudo, é gerado um comportamento global emergente a partir das pequenas e simples interações locais entre os indivíduos do sistema. Os nós sensores terrestres são configurados para acionar alarmes na ocorrência de entrada de um intruso na área, enquanto os UAVs recebem os alarmes e têm que decidir qual UAV é o mais hábil a tratar o alarme acionado.
}
{
This work presents an investigation of strategies to coordinate a set of static ground sensor nodes (deployed on the ground) and Unmanned Aerial Vehicles (UAVs) carrying a variety of sensors. This coordination aims to provide efficient surveillance and intrusion detection in a given area of interest. To develop this coordination has been used techniques of Emergent Self-Organization. These techniques do not use external or centralized controls, however, generate an emerging global behavior from the small and simple local interactions ammong the individuals of the system. The ground nodes are setup to trigger alarms in the event of an intruder entrance in the area, while the UAVs receive the alarms and must decide which one is the most skilled to handle the received alarm.
}                        


% Este trabalho apresenta uma investigação de estratégias para coordenar um conjunto de nós sensores terrestres estáticos (posicionados no chão) e de Veículos Aéreos Não Tripulados (UAV) que carregam uma variedade de sensores. Esta coordenação tem como objetivo prover monitoramento e detecção eficientes de intrusos em uma determinada área de interesse. Dentre as estratégias utilizadas, destacam-se as técnicas de Auto-Organização Emergente. Estas, que se apresentam como técnicas que não  utilizam controles externos ou centrais, ou seja, as entidades individuais interagem entre si localmente; porém, pelas interações locais, promovem um comportamento global emergente. Os nós sensores terrestres são configurados para acionar alarmes na ocorrência de um dado evento de interesse, isto é, entrada de um intruso na área, enquanto os VANTs recebem os alarmes e têm que decidir qual UAV é o mais hábil a tratar o alarme acionado.